%%%%%%%%%%%%%%%%%%%%%%%%%%%%%%%%%%%%%%%%%%%%%%%%%%%%
%%%                                              %%%
%%%           PACKAGES                           %%%
%%%                                              %%%
%%%%%%%%%%%%%%%%%%%%%%%%%%%%%%%%%%%%%%%%%%%%%%%%%%%% 
\CTEXsetup[format={\Large\bfseries}]{section}       % 防止标题的居中
\usepackage{fancyhdr}     %设置页眉页脚的宏包
\usepackage{lastpage}     %引用最后一页


\usepackage[top=2cm, bottom=2cm, left=3cm, right=2cm]{geometry}
\usepackage{ctex}%中文
\usepackage{hyperref}%添加目录
	\hypersetup{colorlinks =true,pdfstartview=Fit,breaklinks=true}
\usepackage{amsmath, amssymb}%数学环境
\usepackage{graphicx}%图片
\usepackage{graphics}%图片
\usepackage{tabularx}%表格
\usepackage{float}%[H]强制固定图片、表格等当前位置
\usepackage{lipsum}%填充一段系统中的文字,用以体验某些模块
\usepackage{multicol}%分栏控制
\usepackage{fancyvrb}%盒子
\usepackage{bm}%加粗
\usepackage{tikz,tikz-cd}
\usepackage{wrapfig}%图片控制
\DeclareMathAlphabet{\mathscr}{OT1}{pzc}{m}{it}%字母字体mathscr
%%伪代码
\usepackage{algorithm}
\usepackage{algorithmicx}
\usepackage{algpseudocode}
\floatname{algorithm}{算法}
\renewcommand{\algorithmicrequire}{\textbf{输入:}}
\renewcommand{\algorithmicensure}{\textbf{输出:}}


%%%%%%%%%%%%%%%%%%%%%%%%%%%%%%%%%%%%%%%%%%%%%%%%%%%%
%%%                                              %%%
%%%          		定理                           %%%
%%%                                              %%%
%%%%%%%%%%%%%%%%%%%%%%%%%%%%%%%%%%%%%%%%%%%%%%%%%%%% 
\usepackage{tcolorbox}%绘制定理环境框框
\tcbuselibrary{many}
\definecolor{mygrey}{RGB}{140,150,160}%定义颜色
% \newtcbtheorem[方括号中是编号的设置]{定理环境名\begin{}\end{}的括号中填入的名字}{pdf显示出来的名字}
%{......
%label separator=* 默认冒号:}
%{thm 引用名前缀,配合前面label separator,\ref{}}
%\begin{sometheorem}{My example}{myex}
%My theorem text.
%\end{sometheorem}
%See Example~\ref{thm*myex}.                <----



%定义
\newtcbtheorem[auto counter,number within=section]{definition}{定义}
{fonttitle=\bfseries\upshape, fontupper=\slshape,
     arc=1mm, colback=white,colframe=mygrey,label separator={.}}{def}

%prop,lemma,thm,corollary
\newtcbtheorem[auto counter,number within=section]{thm}{定理}{
  enhanced,
  sharp corners,
  attach boxed title to top left={
    yshifttext=-1mm
  },
  colback=white,
  colframe=mygrey,
  label separator={.},
  fonttitle=\bfseries,
  boxed title style={
    sharp corners,
    size=small,
    colback=mygrey,
    colframe=mygrey,
  } 
}{thm}
%use counter from=thm 表示使用和thm一样的计数器
\newtcbtheorem[use counter from=thm]{lemma}{引理}{
  enhanced,sharp corners,attach boxed title to top left={yshifttext=-1mm },
  colback=white,colframe=mygrey,label separator={.},fonttitle=\bfseries,
  boxed title style={sharp corners,size=small,colback=mygrey,colframe=mygrey,} 
}{lem}
\newtcbtheorem[use counter from=thm]{cor}{推论}{
  enhanced,sharp corners,attach boxed title to top left={yshifttext=-1mm },
  colback=white,colframe=mygrey,label separator={.},fonttitle=\bfseries,
  boxed title style={sharp corners,size=small,colback=mygrey,colframe=mygrey,} 
}{cor}
\newtcbtheorem[use counter from=thm]{prop}{命题}{
  enhanced,sharp corners,attach boxed title to top left={yshifttext=-1mm },
  colback=white,colframe=mygrey,label separator={.},fonttitle=\bfseries,
  boxed title style={sharp corners,size=small,colback=mygrey,colframe=mygrey,} 
}{prop}
\newtcbtheorem[use counter from=thm]{axiom}{公理}{
  enhanced,sharp corners,attach boxed title to top left={yshifttext=-1mm },
  colback=white,colframe=mygrey,label separator={.},fonttitle=\bfseries,
  boxed title style={sharp corners,size=small,colback=mygrey,colframe=mygrey,} 
}{axiom}



%%%%%%%%%%%%%%%%%%%%%%%%%%%%%%%%%%%%%%%%%%%%%%%%%%%%
%%%                                              %%%
%%%            证明与解                          %%%
%%%                                              %%%
%%%%%%%%%%%%%%%%%%%%%%%%%%%%%%%%%%%%%%%%%%%%%%%%%%%% 
\usepackage{amsthm}
\theoremstyle{plain}% from `amsthm'
%证明、解
\newenvironment{pf}{{\noindent\it\textbf{ Proof:}}\\}{\hfill $\square$\par}
\newenvironment{sol}{{\noindent\it\textbf{ Sol:}}\\}{\hfill $\square$\par}
\tcolorboxenvironment{pf}{% `proof' from `amsthm'
blanker,breakable,left=5mm,
before skip=10pt,after skip=10pt,
borderline west={1mm}{0pt}{mygrey}}
\tcolorboxenvironment{sol}{% `proof' from `amsthm'
blanker,breakable,left=5mm,
before skip=10pt,after skip=10pt,
borderline west={1mm}{0pt}{mygrey}}



%%%%%%%%%%%%%%%%%%%%%%%%%%%%%%%%%%%%%%%%%%%%%%%%%%%%
%%%                                              %%%
%%%           		 注释                          %%%
%%%                                              %%%
%%%%%%%%%%%%%%%%%%%%%%%%%%%%%%%%%%%%%%%%%%%%%%%%%%%% 

\usepackage[framemethod=TikZ]{mdframed}

\newenvironment{rk}[1][]{
	\mdfsetup{
		frametitle={
			\tikz[baseline=(current bounding box.east), outer sep=0pt]
			\node[anchor=east,rectangle,fill=white]
			{\strut Rk.\ifstrempty{#1}{}{~#1}};},
		innertopmargin=10pt,linecolor=mygrey,
		linewidth=2pt,topline=true,
		frametitleaboveskip=\dimexpr-\ht\strutbox\relax
	}
	\begin{mdframed}[]\relax
}{\end{mdframed}}

\newenvironment{nota}[1][]{
	\mdfsetup{
		frametitle={
			\tikz[baseline=(current bounding box.east), outer sep=0pt]
			\node[anchor=east,rectangle,fill=white]
			{\strut Rk.\ifstrempty{#1}{}{~#1}};},
		innertopmargin=10pt,linecolor=mygrey,
		linewidth=2pt,topline=true,
		frametitleaboveskip=\dimexpr-\ht\strutbox\relax
	}
	\begin{mdframed}[]\relax
}{\end{mdframed}}
%上方标题无前缀的盒子
\newenvironment{mybox}[1][]{
	\mdfsetup{
		frametitle={
			\tikz[baseline=(current bounding box.east), outer sep=0pt]
			\node[anchor=east,rectangle,fill=white]
			{\strut \ifstrempty{#1}{}{~#1}};},
		innertopmargin=10pt,linecolor=mygrey,
		linewidth=2pt,topline=true,
		frametitleaboveskip=\dimexpr-\ht\strutbox\relax
	}
	\begin{mdframed}[]\relax
}{\end{mdframed}}


%%%%%%%%%%%%%%%%%%%%%%%%%%%%%%%%%%%%%%%%%%%%%%%%%%%%
%%%                                              %%%
%%%           		  例子                         %%%
%%%                                              %%%
%%%%%%%%%%%%%%%%%%%%%%%%%%%%%%%%%%%%%%%%%%%%%%%%%%%% 
\definecolor{tsyellow}{RGB}{200,200,200}
\newenvironment{boxexample}
{\begin{tcolorbox}
[enhanced jigsaw,breakable,pad at break*=1mm,
 colback=tsyellow!20,boxrule=0pt,frame hidden]
}
{\end{tcolorbox}}


\newtheorem{envexample}{Example}[section]

\newenvironment{ex}
               {\begin{boxexample}\begin{envexample}}
               {\end{envexample}\end{boxexample}}




%%注释掉的内容,以前的定理环境
%\usepackage{verbatim}
%\begin{comment}
%%%定理环境
%\usepackage{ntheorem}%定理去编号
%\newtheorem{definition}{\hspace{2em}定义}[chapter]
%% 如果没有章, 只有节, 把上面的[chapter]改成[section] 
%\newtheorem{thm}[definition]{\hspace{2em}定理}
%\newtheorem{axiom}[definition]{\hspace{2em}Axiom} 
%\newtheorem{lemma}[definition]{\hspace{2em}引理} 
%\newtheorem{prop}[definition]{\hspace{2em}Proposition}
%\newtheorem{cor}[definition]{\hspace{2em}推论} 
%%前面几项共享编号,下面证明、注释与前面分开编号或者不编号
%\newtheorem*{pf}{\hspace{2em}Proof}[section] 
%\newtheorem*{rk}{\hspace{2em}Rk.}[section]
%\newtheorem*{nota}{\hspace{2em}Notatoin.}[section]
%\end{comment}


%%%%%%%%%%%%%%%%%%%%%%%%%%%%%%%%%%%%%%%%%%%%%%%%%%%%
%%%                                              %%%
%%%           		   编号                        %%%
%%%                                              %%%
%%%%%%%%%%%%%%%%%%%%%%%%%%%%%%%%%%%%%%%%%%%%%%%%%%%% 




%%%%代码%%%%
%导入listings宏包
\RequirePackage{listings}
 
%先自定义三种颜色
\definecolor{dkgreen}{rgb}{0,0.6,0}
\definecolor{gray}{rgb}{0.5,0.5,0.5}
\definecolor{mauve}{rgb}{0.58,0,0.82}
\definecolor{CPPViolet} {HTML} {7040A0}

\lstset{
    basicstyle          =   \sffamily,          % 基本代码风格
    keywordstyle        =   \bfseries,          % 关键字风格
    commentstyle        =   \rmfamily\itshape,  % 注释的风格,斜体
    stringstyle         =   \ttfamily,  % 字符串风格
    flexiblecolumns,                % 别问为什么,加上这个
    numbers             =   left,   % 行号的位置在左边
    showspaces          =   false,  % 是否显示空格,显示了有点乱,所以不现实了
    numberstyle         =   \zihao{-5}\ttfamily,    % 行号的样式,小五号,tt等宽字体
    showstringspaces    =   false,
    captionpos          =   t,      % 这段代码的名字所呈现的位置,t指的是top上面
    frame               =   lrtb,   % 显示边框
    language        =   python, 
    basicstyle      =   \zihao{-5}\ttfamily,
		 numberstyle=\tiny\color{gray},
    keywordstyle    =   \color{blue},
%    identifierstyle = \color{purple!50},
    stringstyle     =   \color{magenta},
    commentstyle    =   \color{dkgreen}\ttfamily,
    breaklines      =   true,   % 自动换行,建议不要写太长的行
    columns         =   fixed,  % 如果不加这一句,字间距就不固定,很丑,必须加
    basewidth       =   0.5em,
    morekeywords={alignas,continute,friend,register,true,alignof,decltype,goto,
    reinterpret_cast,try,asm,defult,if,return,typedef,auto,delete,inline,short,
    typeid,bool,do,int,signed,typename,break,double,long,sizeof,union,case,
    dynamic_cast,mutable,static,unsigned,catch,else,namespace,static_assert,using,
    char,enum,new,static_cast,virtual,char16_t,char32_t,explict,noexcept,struct,
    void,export,nullptr,switch,volatile,class,extern,operator,template,wchar_t,
    const,false,private,this,while,constexpr,float,protected,thread_local,
    const_cast,for,public,throw,std, UINT, BYTE},
    emph={map,set,multimap,multiset,unordered_map,unordered_set,
    unordered_multiset,unordered_multimap,vector,string,list,deque,
    array,stack,forwared_list,iostream,memory,shared_ptr,unique_ptr,
    random,bitset,ostream,istream,cout,cin,endl,move,default_random_engine,
    uniform_int_distribution,iterator,algorithm,functional,bing,numeric,
     queue, fstream, graphics.h, easyx.h, conio.h, cstdlib, cstring, string, ctime, 
    queue, assert.h, algorithm, sstream, strstream, iomanip},
    emphstyle=\color{CPPViolet}, 
    escapechar = @,
}
